\section{Discussion}
\label{s:discussion}
\label{s:disc}

\subsection{\Sys and \git submodule}
\Git submodule\HP{TODO: cite Chapter 6.6 in git book} is a tool in \git which
allows a user to include a project (repository) into another. The project to be included
is called a submodule. \Git will consider the submodule a special object in the
object tree in further commit. This special object contains the commit id of the
submodule repository. The detail update of files in the submodule repository is
contained in the commits in the submodule instead of the parent repository.

Since \git submodule manages the submodule repository separately from the parent repository,
it is possible to separate access control in \git submodule. The reason is that
\git can set different permissions for the submodule repository and parent
repository such that a user can visit the submodule repository but not the parent repository,
or vice versa. \Git submodule can also have similar functionalities with partial
checkout and partial commit, because a user can checkout or commit only to the
submodule repository, instead of the whole parent project.

\mw{Whoah. Not a good idea to say things like ``X doesn't work for our
problem.'' (One must be careful.) Instead, better just to state the
shortcomings of the alternative. See below for edits.}

However, 
%\git submodule is not a great solution for our motivating problems.
%One of the reasons is that
each submodule is actually an independent \git
repository, making it impossible, or at least very difficult, to share a subset
of a repository. 
Also,
%Another deficiency for
git submodule
%is that it
does not
provide a hierarchical structure, making it difficult to manage submodules
%when the repository is becoming huge.
in a large repository.
Finally, \git submodule can only separate %the
access permissions by space, but not by time: a user with access to a submodule
repository still has
%the permissions to visit all the history of it.
access to the entire history.

\iffalse
\subsection{\Sys and \git access management tools}
%\textred{Talk about gitosis and gitolite, refer to Chapter 4 in git book.}
\fi

\subsection{\Sys and \git subtree merge}
%\textred{Talk about git subtree merge. The latex comment has some discussions.}

\Git subtree merge is a kind of mechanism for \git merge.
\cite{git-subtree-merge}\HP{cite chapter 6.7 in git book}
The mechanism allows a user to combine two
commit trees from different branches together to form a new commit tree.
The basic usage of the mechanism is to merge two different projects into one.

It has some similarity with "partial commit" in FVM, because what partial commit
does is to merge the changes of the user's work dir (or a subtree) with the
original commit tree. However there are differences. git subtree merge can only
be applied to different branches, and as the document says, are used to merge
two different projects together. Instead, FVM partial commit is used to combine
the subset of changes with the (unchanged files) in old commit. Another
difference is that in subtree merge, one of the tree becomes the subtree of
another, but in FVM partial commit, a subset files replaces the same position in the old commit.


