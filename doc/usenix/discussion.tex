\section{Discussion}
\label{s:discussion}
\label{s:disc}

\subsection{\Sys and \git submodule}
%\textred{Talk about git submodule.}

\subsection{\Sys and \git access management tools}
%\textred{Talk about gitosis and gitolite, refer to Chapter 4 in git book.}
\iffalse
\fi

\subsection{\Sys and \git subtree merge}
%\textred{Talk about git subtree merge. The latex comment has some discussions.}

\Git subtree merge is a kind of mechanism for \git merge. \hw{More word for git
subtree merge.}
\cite{git-subtree-merge}

\iffalse
It seems to be a mechanism for "git merge". The mechanism is to combine two
commit trees from different branches together to form a new commit tree.

It has some similarity with "partial commit" in FVM, because what partial commit
does is to merge the changes of the user's work dir (or a subtree) with the
original commit tree. However there are differences. git subtree merge can only
be applied to different branches, and as the document says, are used to merge
two different projects together. Instead, FVM partial commit is used to combine
the subset of changes with the (unchanged files) in old commit. Another
difference is that in subtree merge, one of the tree becomes the subtree of
another, but in FVM partial commit, a subset files (which does NOT need to be a
tree!) replaces the same position in the old commit.

So basically these are two different mechanisms. subtree merge cannot be used to
solve our problems. We may include some discussions in the paper, if needed.
\fi

