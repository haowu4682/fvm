\section{Design of \sys}
\label{s:design}

\begin{figure}[t]
%\centerline{\includegraphics{fig/arch-abstract}}
\caption{Architecture of \sys. TODO: Draw the figure}
\label{f:arch-abstract}
\end{figure}

\endinput




Figure \ref{f:arch-abstract} \hw{TODO graph} shows the architecture of \sys.
\hw{Describe each part in \sys.}
%\hw{One small paragraph to remind the user that we use \git data model basics.}

\subsection{Data Model}

%\hw{copied from design\_v2.txt}
As said before, \sys separates the \emph{repository} and the \emph{working
directory} (\emph{workdir}). A repository is a database which stores files
and directories of a file system. The repository is stored on the server
machine. A workdir is a directory in a file system which is used to
contain a subset of files from a repository, and allows the user to view and
modify. (A user cannot view and modify files from a repository directly.) The
workdir belongs to the client machine.

\hw{TODO Briefly introduce commit, tree and blob.}

\hw{TODO Include a figure showing the relation between repository and workdir:
notes/gv/partial.dot.pdf.}

% Partial checkout/commit stuff
Figure \ref{f:partial-data-model} shows the relation between the workdir and the
corresponding commit in the repository. \hw{TODO: explain the figure.}

\subsection{Access Control Model}

The motivation scenarios suggest that \sys should have an access control method
throughout both the time and the space. \hw{Check with section 2.}
Therefore, \sys uses a two dimensional \HP{Any good word here?} way to record
access control information. One dimension is the access permissions in each
file, the other dimension is an access control list in each commit.

The access permissions\HP{Is this phrase good?} in each file is \unix-like.
\hw{TODO Describe the access control information for each file.}

\hw{TODO Describe access control list.}
\hw{TODO Describe how access control model affects version control actions.}

\subsection{Branch Mode Model}

\hw{Describe the branch mode basics.}
\hw{Describe how files in branch mode are mapped in the repository.}
\hw{Describe how to enter/exit branch mode.}

\subsection{Trace Level and Automatical Commit}
\hw{Mention trace level and automatical commit stuff. This subsection should be
relatively short.}

