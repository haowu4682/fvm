\section{Introduction}
\label{s:intro}

Today, most popular \HP{Is there a better word here?} \emph{version control
systems (VCS)}\HP{Explain what is an VCS? I think the reader should have the
context.} make two implicit assumptions. First, a repository is considered to be
an integrity, and a user either can access everything in the repository, or
nothing at all. Second, the working directory of a user is a mirror to the
repository: it loads a snapshot from the repository, and stores another snapshot
into the repository after the user's modifications. \hw{How to explain this one?
I want to say: the working directory contains the whole directory tree instead
of a subset of it.}

These assumptions limit the functionality an VCS could provide.
One problem is that an VCS under these assumptions can not allow a user to
visit only a \emph{subset} of a repository. Section \ref{s:motivation} describes
more scenarios for the problems.\HP{Shall we mention detailed examples here? If we
mention a detail scenario, it may be redundant since we already have section
\ref{s:motivation} to talk about motivation scenarios. But if we do not mention
a detail scenario, readers might not be easy to see the problems.}

\hw{TODO: Explain git\cite{git} as an example, if we do not mention detail
scenario above.}

\hw{Shall we introduce some traditional VCS here?}

\hw{Is the length of following 3 paragraphs appropriate?}
We present the design, implementation and evaluation of \sys, a version control
system which solves the issues we mentioned previously. As depicted in Figure
\hw{Fill in the figure, use notes/gv/server.dot.pdf as a start.}, \sys basically
uses a server-client model. The server stores the repository which multiple
users share, and is trusted\HP{any better word?} for accessing and modifying anything in the
repository. The client is the machine which a user works on, and has only
limited permissions to access and modify the repository. \sys assumes that the
server and client share their file systems using a distributed file system, like
\nfs. % This assumption's purpose is to reduce network cost for transmitting files.
When the client wants to perform any action with the repository, it sends the
request to the server, and the server will perform the action with the data in
the shared file system. \hw{Readers might wonder why we do not use a
fully distributed model. We may need to explain it somewhere.}

In order for the user to checkout and commit to the repository while only
accessible to a subset of the repository, \sys has a mechanism called partial
checkout and partial commit. The user could only checkout a subset of files in
the repository, and commit only those files back to the repository when the user
has modified it.

\sys uses \git as its background object storage. However, in order to record the
access control information, we modify the data model a little bit, and add an
access control list into each commit. The server is responsible to check the
access control list each time it performs an action to make sure that the user
has the ability to perform the action.

\hw{We need to mention branch mode, but it exceeds the limitation of 3
paragraphs. What shall we do here?}

A user could use branch mode in \sys similar to popular version control systems.
Unlike normal VCS, a user could specify a subset of files to be commited into a
specified branch. In this way, a user can work with different files in different
branch without duplicating a work directory. \hw{Motivation Case \#4 is an
application of this.}

\hw{Mike, the text flow looks weird here. Do you have suggestions to make it
fluent here?}

During the design and implementation, \sys faces several challenges. First of
all, \sys has to perform access control without affecting version control.
Secondly, \sys has to allow a user to rollback without affecting the mainstream
of a repository. Finally, \hw{the ability to allow different users to modify
concurrently without affecting each other.}

For the first challenge, \sys put all version control actions on the server
side. The server side runs a daemon program which checks access abilities before
doing any actual action. The daemon will hide information that the user should
not read and ignore changes that the user could not writeabilities before doing
any actual action. The daemon will hide information that the user should not
read and ignore changes that the user could not write.

For the second challenge, \sys forces the user to use branch mode when the user
checkout a retro commit.\HP{Similar to "detached HEAD" in git} User modification
will not be merged with mainstream\HP{master branch} unless the user explicitly
specified it.

\hw{TODO Response the 3rd challenge.}

In our experiments, \hw{Fill in summary of evaluation.}

\hw{TODO List of contributions.}

The rest of the paper is organized as follows. \hw{Introduce the following
sections.}

