\section{Introduction}
\label{s:intro}

\newcommand{\git} {Git}
\newcommand{\nfs} {NFS}

\hw{Limitations of previous systems.}

\hw{Motivation Samples.}

\hw{Is the length of following 3 paragraphs appropriate?}
We present the design, implementation and evaluation of \sys, a version control
system which solves the issues we mentioned previously. As depicted in Figure
\hw{Fill in the figure, use notes/gv/server.dot.pdf as a start.}, \sys basically
uses a server-client model. The server stores the repository which multiple
users share, and is trusted\HP{any better word?} for accessing and modifying anything in the
repository. The client is the machine which a user works on, and has only
limited permissions to access and modify the repository. \sys assumes that the
server and client share their file systems using a distributed file system, like
\nfs. % This assumption's purpose is to reduce network cost for transmitting files.
When the client wants to perform any action with the repository, it sends the
request to the server, and the server will perform the action with the data in
the shared file system. \hw{Readers might wonder why we do not use a
fully distributed model. We may need to explain it somewhere.}

In order for the user to checkout and commit to the repository while only
accessible to a subset of the repository, \sys has a mechanism called partial
checkout and partial commit. The user could only checkout a subset of files in
the repository, and commit only those files back to the repository when the user
has modified it.

\sys uses \git as its background object storage. However, in order to record the
access control information, we modify the data model a little bit, and add an
access control list into each commit. The server is responsible to check the
access control list each time it performs an action to make sure that the user
has the ability to perform the action.

During the design and implementation, \sys faces several challenges. \hw{What
challenges shall be included?}

In our experiments, \hw{Fill in summary of evaluation.}

\hw{List of contributions.}

The rest of the paper is organized as follows. \hw{Introduce the following
sections.}

