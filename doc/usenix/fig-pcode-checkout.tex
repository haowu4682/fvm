\begin{figure}[htb]
%\goodcitationsize
\small
\rule{\linewidth}{.08em}

\begin{tabbing}
xx\=xx\=xx\=xx\=\kill

{\bf function} \va{checkout}(\va{commit\_id}, \va{relative\_path},
\va{destination\_path}, \\
\>\>\> \va{access\_control\_list}, \va{repo}) \\
\> root\_tree = \va{get\_commit\_tree}(\va{commit\_id}, \va{relative\_path},
\va{repo}) \\
\> \va{write\_object}(\va{root\_tree}, \va{destination\_path},
\va{access\_control\_list}, \va{repo}) \\ [2mm]

{\bf function} \va{get\_commit\_tree} (\va{commit\_id}, \va{path}, \va{repo}) \\
\> commit = \va{get\_commit}(\va{repo}, \va{commit\_id}) \\
\> root\_tree = \va{get\_commit\_tree}(commit) \\
\> path\_list = \va{split}(path) \\ % HW: Explain what "split" means.
\> {\bf for} entry {\bf in} path\_list \\
\>\> root\_tree = \va{get\_subtree}(root\_tree, entry) \\ [2mm]

// TODO more code from notes/pcode.txt

\end{tabbing}

\rule{\linewidth}{.08em}
\caption{Pseudocode for partial checkout and commit.}
\label{f:pcode-checkout}
\end{figure}

\endinput

% Obsoleted representation
\begin{algorithmic}
    \Function{checkout}{$\mathit{commit\_id}, \mathit{relative\_root\_path},
        \mathit{destination\_path}, \mathit{access\_control\_list}, \mathit{repo}$}
    \EndFunction
\end{algorithmic}

