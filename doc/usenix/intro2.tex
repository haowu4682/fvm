\section{Introduction}
\label{s:intro}

Version control systems (VCS) are used to maintain changes of user data,
including code, document, multimedia data, etc.\HP{How to say the paper revisits
the functionality of VCS? I could not find good sentences here.} Each single
change is usually called a "version", "revision" or "commit" in different
systems. There are many ways to manage commits, but the most commonly used
method is to keep them in a directed acyclic graph (DAG). Such a graph is called
a commit history. Each commit in the commit history contains a directory tree,
which represents the current status of the changes of the recorded data. The
database which contains the whole commit history is often called a repository.

In pratice, a repository is often shared by multiple users. The owner of a
repository may need to restrict the access permissions for specified users. For
example, a project manager might prevent one of his newly joined developer from
reading ancient commits in the commit history. An example is that a web designer
might have no permission to read the backend database code.

Another important functionality for VCS is branching. Different systems may have
different definitions of branches. In this paper, we consider a branch as a
directed path from the commit history graph. A branch is often used to separate
the changes of a subset of files such that those changes are temporarily
independent from the other changes.\HP{Or we can say "the commits have no
dependencies"} A user may need to commit different files to commits of different
branches. For example, \hw{Mention motivation case \#4 here.}

Most systems manage the access control in a repository-based way. In this way,
the administrator needs to create different repositories for different access
permissions. For example, an administrator using GitHub\cite{github} can assign
the permission of a user to read-only, or both read and write. However, the
administrator cannot allow a user to only able to modify a subset of files. If
the administrator wants to do that, he or she must create a new project for the
subset of files, and assign different permissions to that. This is costly and
hard to manage.

For the branch problem, it is possible but complicated for a user to maintain
multiple branches. One way for a user to do that is to maintain multiple copies
of a repository, while each copy has one of the branches. Another way is to
frequently performs "checkout" command on different branches.

One of the reasons for the problems is that VCSs today often
make two implicit assumptions. First, a repository is considered to be
an integrity, and a user either can access everything in the repository, or
nothing at all. Second, the working directory of a user is a mirror to the
repository: it loads a snapshot from the repository, and stores another snapshot
into the repository after the user's modifications. \HP{How to explain this one?
I want to say: the working directory contains the whole directory tree instead
of a subset of it.}

We present the design, implementation and evaluation of \sys, a version control
system which solves the issues we mentioned previously. As depicted in Figure
\hw{Fill in the figure, use notes/gv/server.dot.pdf as a start.}, \sys basically
uses a server-client model. The server stores the repository which multiple
users share, and is trusted\HP{any better word?} for accessing and modifying anything in the
repository. The client is the machine which a user works on, and has only
limited permissions to access and modify the repository. \sys assumes that the
server and client share their file systems using a distributed file system, like
\nfs. % This assumption's purpose is to reduce network cost for transmitting files.
When the client wants to perform any action with the repository, it sends the
request to the server, and the server will perform the action with the data in
the shared file system. \hw{Readers might wonder why we do not use a
fully distributed model. We may need to explain it somewhere.}
\HP{As Mike mentioned, the paragraph started from the figure is unclear. We'll
revisit it after we have the graph.}

In order for the user to checkout and commit to the repository while only
accessible to a subset of the repository, \sys has a mechanism called partial
checkout and partial commit. The user could only checkout a subset of files in
the repository, and commit only those files back to the repository when the user
has modified it. The basic idea for partial checkout is to copy the specified
subset of files to the work directory, and the idea for partial commit is to
merge the changes of the specified subset of files with the parent commit tree.

\sys uses \git as its background object storage. However, in order to record the
access control information, we modify the data model by adding an access control
list into each commit, and adding access permission information into each blob
and tree. The server is responsible to check the
access control list each time it performs an action to make sure that the user
has the ability to perform the action.

\sys provides a special branch mode for the users to manage branches. \Sys
maintains a mapping from files to branch names. During the commit process, the
change of a file only goes to the branch in the mapping, and the other branches
still have the old version of the file.
\hw{Consider the following sentence: \sys requires a checkout of previous commit
to be done in branch mode because it wants to avoid the "detached head" state in
\git.}

A user could use branch mode in \sys similar to popular version control systems.
Unlike normal VCS, a user could specify a subset of files to be commited into a
specified branch. In this way, a user can work with different files in different
branch without duplicating a work directory. \hw{Motivation Case \#4 is an
application of this.}

During the design and implementation, \sys faces several challenges. First of
all, \sys has to perform access control without affecting version control.
Secondly, \sys has to allow a user to rollback without affecting the mainstream
of a repository. Finally, \hw{the ability to allow different users to modify
concurrently without affecting each other.}

For the first challenge, \sys assumes the server side could always be trusted.
This is reasonable because the owner of a repository is often the person who
chooses a server, and the owner is trusted to access the repository he or she creates.
Under the assumption, the server side runs a daemon program which checks access abilities before
doing any actual action. The daemon will hide information that the user should
not read and ignore changes that the user could not write before doing
any actual action. The daemon will hide information that the user should not
read and ignore changes that the user could not write.

For the second challenge, \sys forces the user to use branch mode when the user
checkout a retro commit.\HP{Similar to "detached HEAD" in git} User modification
will not be merged with mainstream\HP{master branch} unless the user explicitly
specified it.

\hw{TODO Response the 3rd challenge.}

In our experiments, \hw{Fill in summary of evaluation.}

\hw{TODO List of contributions.}

The rest of the paper is organized as follows. \hw{Introduce the following
sections.}

