\section{Related work}
\label{s:related}
\label{s:relwork}

\subsection{Client-Server Model VCS}
We start by reviewing VCSs with client-server model. In this model, the server
stores a repository, and the client needs to communicate with the server to
perform actions. In some systems, the client could do some work locally, but
basically it cannot work as a standalone piece.

Revision Control System (RCS)\cite{tichy82rcs} is used to maintain versions for a single file. It
has the basic functionalities such as creating, logging and merging versions.
Its disadvantage point is that it is designed to maintain versions for each
single file. It does not provide a clear interface for combining different files
into a repository.

Concurrent Versions System (CVS)\cite{cvs} is a client-server model version control
system. It improves RCS and supports grouping of files in a repository. It also
allows different users to share a repository. However there are drawbacks for
CVS. The client must connect to the server to perform normal version control
actions. Also there is not a global state of versions - each file still has its
own versioning.

Apache Subversion (SVN)\cite{svn} is a successor of CVS. SVN has a similar data model and
user's point of view with CVS. It allows more functions to be executed locally
on the client side. However, it is still a client-server model, making the main
problems of CVS still exist in SVN.

ClearCase\cite{clearcase} is a software configuration management (SCM) system. It provides versioning
by using dynamic views. A view is a method to map normal files into versioned
files. The view provides a way to automatically group file versions. However,
views in ClearCase are often just snapshot views which is just similar to a
regular commitment, or dynamic views which map into the most recent versions of
files.

Vesta\cite{vesta} is a a SCM system created by Compaq Inc. Its basic data elemement is
package. A package is a group of related files. The packages are in a
hierarchical structure.

Vesta's mechanisms for checkout and commit are also based on packages, so it
provides a mechanism similar to partial checkout and partial commit with \sys.
A user can checkout a package to client side, modify it and checkin it into the
repository.

However Vesta still has its deficiencies. Vesta does not provide concurrency
modification. If a user has already checked out a version of a package, other
user who wants to checks out the same version cannot check in modifications of the
same package. The branch specification for Vesta is awkward, and it only
supports branching for each package instead of the whole repository. Finally it
has no solution for access control problem.

\subsection{Distributed Model VCS}

In a distributed model VCS, each client contains a copy of the whole repository.
Each client can work as a standalone piece, while they could still communicate
with each other. There are many existing systems with such a kind of design.
GNU Arch\cite{gnuarch}, Monotone\cite{monotone}, Darcs\cite{darcs},
BitKeeper\cite{bitkeeper}, Mercurial\cite{mercurial} and Git\cite{git} are
several representative systems.

Mercurial is designed to be an alternative to BitKeeper. It stores file deltas
in data storage. It uses data model called revlogs\cite{mackall06mercurial} to
retrieve full files from the data. Its user interface is similar with Git.

Git\cite{git} is a popular revision control tool. Unlike normal tools, Git
stores every single version of file as an independent blob. Each commit is a
hierarchical structure of blobs. This kind of data model makes it very easy to
checkout a previous commit and to create a new branch. However it makes a
repository containing a lot of redundant information for files.

% Time machine etc

% Versioned file systems

%\subsection{}

