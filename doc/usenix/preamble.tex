% This is stuff that is required for the document to compile. It is not
% related to formatting. 
\usepackage[pdftex]{graphicx}
\usepackage{subfigure}
\usepackage{booktabs}  % for tables
\usepackage{dcolumn}   % for tables
%\usepackage{hhline}   % for tables
\usepackage{multirow}  % for tables
\usepackage{rotating}
\usepackage{xspace}
\usepackage{hyphenat}  % supplies \hyp{}, which tells tex that it can 
		       % hyphenate at an existing hyphen
\usepackage{color}
\usepackage{enumerate}

\usepackage{textcomp}
\usepackage{amsmath}
\usepackage{amssymb}
\usepackage{amsfonts}
\usepackage{lastpage}
\usepackage{tabularx}
\usepackage{pifont}

\usepackage{ulem}
\normalem

\usepackage[table]{xcolor}
\usepackage{colortbl} % colorful columns and rows
\usepackage{array}    % needed for 'b' argument in tabular preamble

\definecolor{lightgray}{rgb}{.9,.9,.9}
\definecolor{darkgray}{rgb}{.4,.4,.4}

%\usepackage[margin=0.99in]{geometry}
%\usepackage{algorithm}
%\floatname{algorithm}{Pseudocode}
\usepackage{algpseudocode}
\algrenewcomment[1]{\hfill// #1}%
\algnotext{EndFunction}                % Python-style indentation
\algnotext{EndFor}                     % Python-style indentation

\usepackage[noeka]{mathrmletter}
\usepackage{arydshln}

\newif\ifintrouble
\newif\ifcuttext

%% UNCOMMENT FOLLOWING LINE IN CASE OF EMERGENCY
%\introubletrue
\ifintrouble
\cuttexttrue
\else
\cuttextfalse
\fi

% peanut gallery comments
% NOTE: Comment out the line below if you want a draft with no red comments.
% NOTE: Commenting out this line may replace some of the red comments with 
%       extra spaces or newlines.
%\def\noeditingmarks{}
%

%\newcommand{\textred}[1]{\textcolor{red}{#1}}
\newcommand{\textred}[1]{\begingroup \color{red} #1\endgroup}
\ifx\noeditingmarks\undefined
   \newcommand{\pgwrapper}[2]{\textred{#1: #2}}
   \newcommand{\pgwrapperb}[1]{\textbf{#1}}
\else
   \newcommand{\pgwrapperb}[1]{}
   \newcommand{\pgwrapper}[2]{}
\fi
\newcommand{\hw}[1]{\pgwrapper{HW}{#1}}
\newcommand{\mw}[1]{\pgwrapper{MW}{#1}}
% end peanut gallery comments

% Pseudo code commands
\newcommand{\va}[1]{\textit{#1}}

\newcommand{\annoyingsize}{\fontsize{7.5}{9}\selectfont}
\newcommand{\moreannoyingsize}{\fontsize{7.3}{8.8}\selectfont}
% definitions for MP
\def\hn{\usefont{OT1}{phv}{mc}{n}\selectfont}
\def\hb{\usefont{OT1}{phv}{bc}{n}\selectfont}
\newcommand{\mpfont}{\hn\scriptsize}
\newcommand{\MPworker}[1]{{\color{red}\vrule\vrule}{\marginpar{\color{red}\mpfont #1}}}
\ifx\noeditingmarks\undefined
    \newcommand{\HP}[1]{\MPworker{HW: #1}}
    \newcommand{\MP}[1]{\MPworker{MW: #1}}
\else
   \newcommand{\HP}[1]{}
   \newcommand{\MP}[1]{}
\fi
\setlength{\marginparwidth}{19mm}
\setlength{\marginparsep}{0.35mm}

\ifx\noeditingmarks\undefined
    \newcommand{\changebars}[2]{%
    [{\em \begingroup #1 \endgroup}][\sout{#2}]}
\else
    \newcommand{\changebars}[2]{#1}
\fi

\newcommand{\theSection}[1]{\S\ref{#1}}
%\newcommand{\theSection}[1]{Section~\ref{#1}}
\newcommand{\techReportOnly}[1]{}

\newcommand{\circledone}{\ding{192}\xspace}
\newcommand{\circledtwo}{\ding{193}\xspace}
\newcommand{\circledthree}{\ding{194}\xspace}
\newcommand{\circledfour}{\ding{195}\xspace}
\newcommand{\circledfive}{\ding{196}\xspace}
\newcommand{\filledone}{\ding{202}\xspace}
\newcommand{\filledtwo}{\ding{203}\xspace}
\newcommand{\filledthree}{\ding{204}\xspace}
\newcommand{\filledfour}{\ding{205}\xspace}
\newcommand{\filledfive}{\ding{206}\xspace}

\newcommand{\ie}{i.e.}
\newcommand{\eg}{e.g.}
\newcommand{\ea}{et al.}
\newcommand{\eans}{et al.}

\newcommand{\sys}{\textsc{FVM}\xspace}
\newcommand{\Sys}{\sys}

%\newcommand{\vcs}{VCS\xspace}
%\newcommand{\Vcs}{\vcs}

\newcommand{\git}{\textsc{Git}\xspace}
\newcommand{\Git}{\git}
\newcommand{\nfs}{\textsc{NFS}\xspace}
\newcommand{\Nfs}{\nfs}
\newcommand{\unix}{\textsc{UNIX}\xspace}
\newcommand{\Unix}{\nfs}


%\def\compactify{\leftmargin=\parindent \itemsep=0.01pt \topsep=0.01pt \partopsep=0pt \parsep=0.01pt}
\def\compactify{\itemsep0in \topsep2pt \parsep=0.00in \partopsep=0pt \leftmargin4em}
\let\latexusecounter=\usecounter
\newenvironment{CompactItemize}
  {\def\usecounter{\compactify\latexusecounter}
   \begin{itemize}}
  {\end{itemize}\let\usecounter=\latexusecounter}
\newenvironment{CompactEnumerate}
  {\def\usecounter{\compactify\latexusecounter}
   \begin{enumerate}}
  {\end{enumerate}\let\usecounter=\latexusecounter}
\newenvironment{myenumerate}
  {\def\usecounter{\compactify\latexusecounter}
   \begin{enumerate}}
  {\end{enumerate}\let\usecounter=\latexusecounter}

\def\compactparspace{\itemsep=0ex \topsep1ex \parsep=1ex \partopsep=0pt
\leftmargin\parindent}
\newenvironment{myenumerate2}
  {\def\usecounter{\compactparspace\latexusecounter}
   \begin{enumerate}}
  {\end{enumerate}\let\usecounter=\latexusecounter}


\def\compactforsummary{\itemsep0pt \topsep0pt \parsep=0ex
\partopsep=0pt \leftmargin\parindent}
\newenvironment{myenumerate3}
  {\def\usecounter{\compactforsummary\latexusecounter}
   \begin{enumerate}}
  {\end{enumerate}\let\usecounter=\latexusecounter}



\newenvironment{myitemize}%
  {\begin{list}{\labelitemi}{\itemsep0in \topsep2pt \parsep0.00in
  \partopsep=0pt \leftmargin\parindent}}%
  {\end{list}}

\newenvironment{myitemize2}%
  {\begin{list}{\labelitemi}{\itemsep6pt \topsep6pt \parsep0.00in
  \partopsep=0pt \leftmargin\parindent}}%
  {\end{list}}

\newenvironment{myitemize3}%
  {\begin{list}{\labelitemi}{\itemsep3pt \topsep3pt \parsep0.00in
  \partopsep=0pt \leftmargin\parindent}}%
  {\end{list}}


\newcommand{\astskip}{\smallskip\noindent\parbox{\linewidth}
			{\center*\hspace{2.5em}*\hspace{2.5em}*\medskip\smallskip}}

\def\discretionaryslash{\discretionary{/}{}{/}}
{\catcode`\/\active
\gdef\URLprepare{\catcode`\/\active\let/\discretionaryslash
        \def~{\char`\~}}}%
\def\URL{\bgroup\URLprepare\realURL}%
\def\realURL#1{\tt #1\egroup}%

\newcommand{\tmidrule}{\midrule[0.02em]}

%\renewcommand{\dbltopfraction}{0.1}

\newcommand{\defterm}[1]{\emph{\textbf{#1}}}

% Local Variables:
% tex-command: "gmake;:"
% tex-main-file: "cic10.ltx"
% tex-dvi-view-command: "gmake preview;:"
% End:
